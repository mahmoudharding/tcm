\documentclass [12pt, letterpaper, titlepage] {article}
\usepackage[utf8]{inputenc}
\usepackage{color}
\usepackage{booktabs}
\usepackage[shortlabels]{enumitem} 
\usepackage{amsmath}
\usepackage{hyperref}

%\definecolor{mediumblue}{rgb}{0.0, 0.0, 0.8}

\hypersetup{
    colorlinks=true,
    linkcolor=blue,
    filecolor=magenta,      
    urlcolor=blue,
}

\begin{document}

\begin{titlepage}
\vspace*{3cm}
	\begin{center}
		\Huge {Euler's Method \\ and the \\ Rocket Challenge \\ \vspace{1cm} Teaching Contemporary Mathematics \\ \vspace{1cm}}
		\Large {Mahmoud Harding \\ mahmoud.harding@ncssm.edu \vspace{1cm}}
	\end{center}
\end{titlepage}
\section*{Activity Introduction}
Introduce students to the idea of model rockets using the YouTube video \href{https://youtu.be/WpqUOW19aJQ}{\textbf{Dude Perfect Rocket Challenge}} or another video you prefer.
\begin{enumerate}
	\item{After watching the video ask the students the following questions.}
		\begin{enumerate}[a.]
			\item{What did you notice about the path of the rockets?}
			\item{What do you wonder about the path of the rockets?}
		\end{enumerate}
	\item{Ask the students to sketch a graph that could model the path of a rocket from lift-off to landing.}
	\item{Ask the students to list some factors that affect the height of the rocket.}
\end{enumerate}
\section*{Model Rocket Engine Information}
Reference the \href{https://www.grc.nasa.gov/www/k-12/rocket/rktsafe.html}{\textbf{Model Rocket Guided Tour}} available on the NASA Glenn Research Center website. The \href{https://www.grc.nasa.gov/www/k-12/rocket/rktengine.html}{\textbf{Model Solid Rocket Engine}}, \href{https://www.grc.nasa.gov/www/k-12/rocket/rktengperf.html}{\textbf{Model Rocket Engine Performance}} and \href{https://www.grc.nasa.gov/www/k-12/rocket/rktenglab.html}{\textbf{Model Rocket Engine Designation}} pages of the tour describe how a model rocket engine generates thrust. Consider assigning these pages as preparatory reading. 
	\begin{enumerate}
		\item{Discuss the stages of a model rocket engine burn \textbf{propellant thrust}, \textbf{delay}, and \textbf{ejection}.}
		\item{Ask the students to sketch a graph that could model the thrust of the rocket engine from the time it is ignited until the rocket burns all the propellent.}
                 \item{Show the animation of the rocket engine and the thrust vs. time curve. Have the students check their graphs against the graph generated by the animation and discuss the similarities and differences.}
                 \item{Discuss why a typical model rocket thrust curve looks the way it does.
                 \begin{enumerate}[a.]
                 	\item{What is happening in the engine to make thrust increase sharply at the beginning?}
                 	\item{What is happening in the engine when it peaks and declines?}
                 	\item{What is happening in the engine when the thrust is constant?}
                 	\item{What is happening in the engine when thrust is zero?}
		\end{enumerate}
		}	
	\end{enumerate}
\section*{What is Thrust?}
At this point we want to develop and understanding of the concept of thrust.
\begin{enumerate}
\item{Examine the time/thrust curves from the Estes Rockets Handout. 
			\begin{enumerate}[a.]
				\item{Ask students to determine the units of thrust. Thrust is measured in Newtons \textbf{$$ \text{1N = }\frac{\text{kg} \cdot \text{m}}{\text{sec}^2}.$$}  It is a force that will cause the rocket mass to accelerate.}
				\item{Ask students to examine the line for \textbf{Average Thrust}. Discuss how the average thrust relates to the thrust curve.}
				\item{Ask the students to choose (or randomly assign) two curves from the handout and compare them by noting similarities and differences in the shapes and axes.}
				\item{Ask the students to make conjectures about what we could figure out by using the thrust/time curve.}
			\end{enumerate}
		}
\end{enumerate}
\section*{ESTES B4 Model Rocket}
Give all the students a copy of the National Association of Rocketry Standards and Testing specifications for the ESTES B4 Model Rocket.
\begin{enumerate}
	\item{Ask the students to look at the specification sheet and identify one thing to interpret and two things that they "wonder".}
	\item{Work with the students to identify following items listed on the handout and discuss how they relate to the \textbf{Thrust Curve}.
		\begin{enumerate}[a.]
			\item{Casing Dimensions}
			\item{Peak Thrust}
			\item{Burn Time}
			\item{Total Impulse, which is the force applied over a period of time (duration)}
			\item{Average Thrust}
			\item{Mfg. Recommended Max Liftoff "Weight" (mass of entire rocket)}
			\item{Initial Mass (mass of the engine before ignition)}
			\item{Propellent Mass (mass of the black powder propellant that gives thrust)}
			\item{Mass after firing (mass of the "burnt out" engine)}
			\item{Average measured delay (time between the end of burn and ejection)}	
			\item{$\sigma$, standard deviation of tested measurements}
		\end{enumerate}
	}
	\item{Define the following formulas:
		\begin{enumerate}[a.]
			\item{Impulse is $I = F \cdot \Delta t$, where $I$ is impulse and $F$ is force measured in $\displaystyle \frac{\text{kg} \cdot \text{m}}{\text{sec}^2}$.}
			\item{Momentum is $p = m \cdot v$, where $p$ is momentum, $m$ is the mass and $v$ is the velocity.}
			\item{Impulse is the change in momentum $I = \Delta p = m \cdot \Delta v$}

			\item{Ask the students to estimate the area under the curve to approximate the total impulse (and therefore the change in momentum). Make sure that they pay attention to the units on each axis.  Compare this value to the published \textbf{Total Impulse}. }
			\item{Ask the students to approximate the change in velocity using the estimate for total change in impulse and the mfg. recommended mass liftoff weight, $\displaystyle \frac{\Delta p}{m}=\Delta v$.}
		\end{enumerate}
	}
\end{enumerate}
\newpage
\section*{How High Does the Rocket Go?}
Using the data from the specification sheet (on the back of the handout) we can make a spreadsheet to do a time step analysis to determine the height of the rocket at any point in time, including finding out how high the rocket will go.
\begin{center}
\begin{tabular}{l l c l l}
\toprule
Time (seconds)	 & Thrust (Newtons)&  &  Time (seconds)	& Thrust (Newtons)\\
\toprule
0			& 0 			&	& 0.258			& 4.182 \\ \hline
0.02			& 0.418 		&	& 0.326			& 3.763 \\ \hline
0.04			& 1.673 		&	& 0.422			& 3.554 \\ \hline
0.065		& 4.076 		&	& 0.422			& 3.554 \\ \hline
0.085		& 6.69 		&	& 0.422			& 3.554 \\ \hline
0.105		& 9.304 		&	& 0.549			& 3.345 \\ \hline
0.119		& 11.496 		&	& 0.665			& 3.345 \\ \hline
0.136		& 12.750		&	& 0.776			& 3.345 \\ \hline
0.153		& 11.916 		&	& 0.863			& 3.345 \\ \hline
0.173		& 10.666 		&	& 0.940			& 3.449 \\ \hline
0.187		& 9.304 		&	& 0.991			& 3.349 \\ \hline
0.198		& 7.214		&	& 1.002			& 2.404 \\ \hline
0.207		& 5.645		&	& 1.010			& 1.254 \\ \hline
0.226		& 4.809		&	& 1.030			& 0.000 \\
\bottomrule \\			
\end{tabular}
\end{center}
We will use interval notation to distinguish between the different points in time, $t_i$ and their corresponding values of thrust $T_i$. A time interval will be represented by the "current" time $t_i$ and the "next" time $t_{i+1}$
\begin{enumerate}
	\item{We use a physics formula to relate the thrust to the position/height of the rocket.
		\begin{enumerate}[a.]
			\item{Force is $F=m \cdot a$, where $F$ is the force, $m$ is the mass and $a$ is the acceleration.}
			\item{We have a force due to the thrust of the rocket and weight due to gravity. Ask the students to compare the three mfg. recommended max liftoff weights. (They correspond with different choices of delay). We will make an initial simplifying assumption that the mass of the rocket remains constant from liftoff to landing.}		
		\end{enumerate}
	}
	\item{Change the mass from grams to kg to preserve the units of Newtons, \textbf{1 kg = 1000 grams}.}
	\item{Change the weight to Newtons, $W = m \cdot g$, where $W$ is the weight, $m$ is the mass in kg and $g$ is the acceleration due to gravity.}
	\item{Calculate the net force in Newtons $F_N = T - W$, where $F_N$ is the net force in Newtons, $T$ is the thrust and $W$ is the weight in Newtons.}
	\item{Calculate the acceleration in meters per second squared $\displaystyle \frac{F_N}{m}=a$ where $F_N$ is the net force in Newtons and $m$ is the mass in kg.}
	\item{Ask the students to derive the calculation for velocity based on the fact that acceleration is the change in velocity, $\displaystyle a=\frac{\Delta v}{\Delta t}$.
	\begin{align*}
	a_{i} \cdot \Delta t 		&= \Delta v \\
					 	&= v_{i+1}-v_i \\
	a_{i} \cdot \Delta t + v_i	&= v_{i+1}
	\end{align*}
	To complete the calculation for velocity we need to find the change in time, $\Delta t = t_{i+1}-t_i$. 
	}
	\item{Calculate the change in velocity in meters per second, $a_{i} \cdot \Delta t = \Delta v$.}
	\item{Set the initial velocity $v_0$ in the first row to zero.}
	\item{Calculate the velocity in meters per second, $v_{i+1}=v_i+\Delta v$. for every row \textbf{after} the first one.}
	\item{Ask the students to derive the calculation for position based on the fact that velocity is the change in position, $\displaystyle v=\frac{\Delta h}{\Delta t}$, where $h$ represents the position/height of the rocket.
		\begin{enumerate}[a.]
			\item{Calculate the average velocity over the interval $v_{i}$ in meters per second by using the velocities at time $t_i$ and $t_{i+1}$.}			
			\item{Calculate the height, in meters.
			\begin{align*}
			v_{i} \cdot \Delta t 		&= \Delta h \\
					 			&= h_{i+1}-h_i \\
			v_{i} \cdot \Delta t + h_i	&= h_{i+1}
			\end{align*}
			}
		\end{enumerate}	
	}
	\item{Ask the students to make conjectures about what needs to be done to the calculations to find the maximum height of the rocket.}
	\item{Finally, create data in the spreadsheet to include times after the thrust reaches 0 and extend the formulas to these rows.}
\end{enumerate}
\section*{Interpreting the Model}
Ask students to identify the following (perhaps as a homework assignment). These can be indicated with cell coloring. \begin{enumerate}
	\item{The time and height above the ground at the end of the thrust cycle.}
	\item{The time and height above the ground at the end of the delay (when the parachute would deploy).}
	\item{The time and height above the ground at peak altitude.}
	\item{The time of landing, assuming no parachute is deployed.}
	\item{Maximum vertical speeds in both positive (up) and negative (down) directions.}
\end{enumerate}
\section*{Improving the Model}
\begin{enumerate}
        	\item{Add graphs for height, velocity and acceleration.}
	\item{Adjust for the negative height values in the first rows.}
	\begin{enumerate}[a.]
		\item{Use an IF statement to change acceleration when height is zero or negative}
		\item{Use MAX(0, acceleration) to only zero out negative accelerations}
		\item{Be careful of circular references. Recognize that the average values are 'between' rows.}
	\end{enumerate}
	\item{Adjust for mass of the engine decreasing as it burns.}
	\begin{enumerate}[a.]
		\item{Assume the mass lost is proportional to time or to thrust.}
		\item{Assume the second portion of mass is lost during delay burn (proportional to time or a single eject).}
	\end{enumerate}
	\item{Adjust for "drag" (air resistance) of the rocket and/or deploy a parachute.}
	\begin{enumerate}[a.]
		\item{Drag depends on velocity squared but always opposes the velocity, so use $\text{ABS}\left(v \right)*v$ instead of $v^2$}
		\item{Use an IF statement to change drag area after parachute deployment}
		\item{Ejection time is burn time + delay time.}
		\item{Look for linear "terminal velocity" behavior.}
	\end{enumerate}
\end{enumerate}
\end{document}